\section{Introdução e motivação}

\begin{frame}{}
    \fontsize{14pt}{15.2}\selectfont{
	Apresentação pessoal, integração com a turma, introdução de conceitos básicos de programação de computadores e despertar curiosidade dos alunos sobre o tema.
	
	\vspace{1em}
	}\par
	\vspace{1em}
\end{frame}


\begin{frame}{Aplicação de Avaliação}
    \fontsize{14pt}{15.2}\selectfont{
	Datas importantes
	\vspace{1em}
	}\par
	
	\fontsize{12pt}{15}\selectfont{
	\begin{itemize}%[<+->]  
	    \item xx/xx/2019 - Simulado da Primeira Avaliação (1A)
	    \item \textbf{xx/xx/2019 - Primeira Avaliação}
	    \item xx/xx/2019 - Apresentação dos resultados da 1A
	    \item xx/xx/2019 - Simulado da Segunda Avaliação (2A)
	    \item \textbf{xx/xx/2019 - Segunda Avaliação}
	    \item xx/xx/2019 - Apresentação geral dos resultados
	    \item \textbf{xx/xx/2019 - Segunda Chamada}
	\end{itemize}
	
	}\par
	
	\vspace{1em}
\end{frame}



\begin{frame}{Compromisso semanal}
%     \fontsize{14pt}{15.2}\selectfont{
% 	Datas importantes
% 	\vspace{1em}
% 	}\par
	
	\fontsize{12pt}{15}\selectfont{
	\begin{itemize}%[<+->]  
	    \item Encontros: quintas e sextas
	    \item Período: 08/08/2019 à 20/12/2019
	    \item Início: 19h
	    \item Térmico: 21h30m
	    \item Intervalo: 20h10m até 20h20m (a discutir)
	    \item Sala: 28
	    \item Cada dia são três aulas: 
	        \begin{itemize}%[<+->]  
        	    \item 19h-19h50m
        	    \item 19h50m-20h40m
        	    \item 20h40m-21h30m
        	\end{itemize}
	\end{itemize}
	
	}\par
	
	\vspace{1em}
\end{frame}


\begin{frame}{Metodologia das Aulas}

	\fontsize{11pt}{15}\selectfont{
	\begin{itemize}%[<+->]  
	    \item Resolução de dúvidas gerais: 19h até 19h20m (20min)
	    \item Revisão da aula passada: 19h20m até 19h40m (20min)
	    \item Exercício em sala de aula: 19h40m até 19h50m (10min)
	    \item Adição de novo conteúdo: 19h50m até 21h20m (1h20m)
	    \item Revisão/dúvidas do novo conteúdo: 21h20m até 21h30m (10m)
	\end{itemize}
	}\par
	\vspace{1em}
\end{frame}

\begin{frame}{Metodologia das Avaliação}

	\fontsize{12pt}{15}\selectfont{
	\begin{itemize}%[<+->]  
	    \item Primeira nota: (exercício + simulado + prova)
	        \begin{itemize}%[<+->]  
        	    \item Exercícios (peso +2)
        	    \item Simulado (peso +1)
        	    \item Prova escrita (peso 10)
        	\end{itemize}
    	\item Segunda nota: (exercício + simulado + prova + a definir)
	        \begin{itemize}%[<+->]  
        	    \item Exercícios (peso +1)
        	    \item Simulado (peso +1)
        	    \item Prova escrita (peso a definir)
        	    \item Projeto (peso a definir)
        	\end{itemize}
	\end{itemize}
	}\par
	\vspace{1em}
\end{frame}



\begin{frame}{Google sala de aula}
    \fontsize{14pt}{15.2}\selectfont{
	\vspace{1em}Código da turma \fontsize{34pt}{15.2}\selectfont{92oykpw}
	}\par
	\vspace{0.5em}
	\fontsize{10pt}{12}\selectfont{
	Vamos ter como recurso complementar o Google Sala de Aula [classroom]. Por lá vou deixar o material (slides, referências, atividades, etc.) que vocês vão utilizar durante o curso. Não preocupem-se, também vou deixar uma cópia do material (slides e atividades) disponível em pendrive e levarei comigo durante as aulas para os alunos que não puderem acessar o Google Classroom.
	\vspace{1em}
	
	Como usar o google classroom?

    Use este tutorial \url{https://www.youtube.com/watch?v=l4oSdhLS5fQ} [vídeo] para te ajudar a entender o Google Classroom.

    Clique \url{https://classroom.google.com} para entrar no Google Classroom.

	}\par
\end{frame}


\begin{frame}{Bibliografia básica}
    \fontsize{12pt}{15.2}\selectfont{
	\vspace{1em}
	    \begin{itemize}
            \item SOFFNER, R. Algoritmos e programação em linguagem C. Saraiva, 2013.
            \item MIZRAHI, V. V. Treinamento em linguagem C. 2.ed. São Paulo Pearson.2008.
            \item PEREIRA, S. L. Algoritmos e lógica de programação em C: uma abordagem didática. São Paulo, Érica, 2010.
            \item Mais informações é possível encontrar na ementa.
	    \end{itemize}
	}\par
\end{frame}

