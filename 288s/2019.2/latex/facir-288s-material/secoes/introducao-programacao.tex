\section{Introdução a programação}

\begin{frame}{Estruturas de Programação}
    \fontsize{12pt}{15.2}\selectfont{
	Objetivo principal:
	}
	
    \fontsize{11pt}{15}\selectfont{
	\begin{itemize}
	\vspace{0.5em}
    	\item \textbf{Reconhecer algumas estruturas de programação} - sequências, ciclos (loops), condições, através da exploração e desafios.% em um labirinto.; e
    	\item \textbf{Compreender o conceito de algoritmo} - sequência de instruções.
	\end{itemize}
	}\par
\end{frame}


\begin{frame}{Pensamento Computacional}
    \fontsize{12pt}{15.2}\selectfont{
	
    Vocabulário\\	
	}
	\fontsize{12pt}{13}\selectfont{
	\begin{itemize}
	\vspace{0.5em}
    	\item \textbf{sequência} - ordenação de fatos ou ações;
    	\item \textbf{ciclo (loop)} - sequência de ações, fatos constituintes de um processo periódico que, partindo de um ponto inicial, acabam em um ponto-final que nada mais é que o retorno a esse ponto inicial e consequente recomeço; 
    	\item \textbf{código} - regras usadas para converter instruções ou dados de uma forma para outra; e
    	\item \textbf{condição} - circunstância necessária para que ocorra determinado fato ou situação.
	\end{itemize}
	}\par
	\vspace{1em}
\end{frame}


\begin{frame}{Estruturas de Programação}
    \fontsize{12pt}{15.2}\selectfont{
	Perguntas para a participação da classe:
	}
    \fontsize{11pt}{15}\selectfont{
	\begin{itemize}
	\vspace{0.5em}
    	\item O que significa ``repetir'' um ``bloco'' de código?
    	\item Quando você deve usar um bloco ``se''?
	\end{itemize}
	}\par
	
	\fontsize{12pt}{15.2}\selectfont{
	Discussão com o parceiro de equipe:
	}
    \fontsize{11pt}{15}\selectfont{
	\begin{itemize}
	\vspace{0.5em}
    	\item Você consegue pensar em um motivo pelo qual gostaria de usar um bloco se/senão, em vez de usar apenas o ``se''?
	\end{itemize}
	}\par
\end{frame}



\begin{frame}{Pensamento Computacional}
    \fontsize{12pt}{15.2}\selectfont{
	Objetivo principal:
	}
	
    \fontsize{11pt}{15}\selectfont{
	\begin{itemize}
	\vspace{0.5em}
    	\item apresentar o modelo de ``pensamento computacional'' como uma forma de preparar problemas do mundo real para a representação digital.
	\end{itemize}
	}\par
\end{frame}



\begin{frame}{Pensamento Computacional}
    \fontsize{12pt}{13.2}\selectfont{
	Frequentemente, os cientistas da computação descobrem que são responsáveis por programar soluções para coisas com as quais as pessoas sequer já sonharam -- coisas que nunca foram criadas. Enfrentar um problema que nunca foi solucionado antes pode ser assustador, mas com essas simples ferramentas, tudo é possível.
	\vspace{1cm}
	
	Para ter sucesso nesse tipo de solução, nós vamos praticar um método chamado Pensamento Computacional. O pensamento computacional se baseia em quatro etapas para ajudar a resolver vários tipos de problemas diferentes.
	}
\end{frame}


\begin{frame}{Pensamento Computacional}
    \fontsize{12pt}{13}\selectfont{
	\begin{itemize}
	\vspace{0.5em}
    	\item \textbf{Etapa 1)} Decomposição -- Nós não estamos falando de zumbis! Estamos falando de transformar um problema grande e difícil em algo muito mais simples. Geralmente, problemas grandes são apenas diversos problemas pequenos que foram unidos.
    	
    	\item \textbf{Etapa 2)} Padrões -- Normalmente, quando um problema tem muitas partes menores, você perceberá que essas partes têm algo em comum. Se não tiverem, elas poderão, pelo menos, ter algumas semelhanças evidentes em relação a algumas partes de outro problema solucionado anteriormente. Se conseguir identificar esses padrões, compreender as partes ficará muito mais fácil.
    	
	\end{itemize}
	}\par
\end{frame}


\begin{frame}{Pensamento Computacional}
    \fontsize{12pt}{13}\selectfont{
	\begin{itemize}
	\vspace{0.5em}

    	\item \textbf{Etapa 3)} Abstração -- Depois de reconhecer um padrão, você poderá ``abstrair'' (ignorar) os detalhes que são responsáveis pelas diferenças e usar a estrutura geral para encontrar uma solução que seja válida para mais de um problema.
    	
    	\item \textbf{Etapa 4)} Algoritmo -- Quando sua solução estiver completa, você poderá escrevê-la de um modo que ela possa ser processada passo a passo, para que seja fácil atingir os resultados.
	\end{itemize}
	}\par
\end{frame}



\begin{frame}{Pensamento Computacional}
    \fontsize{12pt}{15.2}\selectfont{
	
    Vocabulário\\	
	}
	\fontsize{11pt}{11}\selectfont{
	\begin{itemize}
	\vspace{0.5em}
    	\item \textbf{Pensamento Computacional} - um método de resolução de problemas que ajuda cientistas da computação a preparar problemas para soluções digitais;
    	\item \textbf{Abstração} - ação de ignorar os detalhes de uma solução de modo que ela possa ser válida para diversos problemas;
    	\item \textbf{Algoritmo} - uma lista de etapas que permitem que você complete uma tarefa;
    	\item \textbf{Decompor} - dividir um problema difícil em problemas menores e mais fáceis;
    	\item \textbf{Padrão} - um tema que se repete diversas vezes; e
    	\item \textbf{Programa} - instruções que podem ser compreendidas e seguidas por uma máquina.
	\end{itemize}
	}\par
	\vspace{1em}
\end{frame}


\begin{frame}{Estruturas de Programação}
    \fontsize{12pt}{15.2}\selectfont{
	Perguntas para a participação da classe:
	}
    \fontsize{11pt}{15}\selectfont{
	\begin{itemize}
	\vspace{0.5em}
    	\item Você consegue nomear alguma das etapas do pensamento computacional?
    	\item Você consegue se lembrar de algum dos padrões que encontramos nos exemplos?
	\end{itemize}
	}\par
	
	\fontsize{12pt}{15.2}\selectfont{
	Discussão com o parceiro de equipe:
	}
    \fontsize{11pt}{15}\selectfont{
	\begin{itemize}
	\vspace{0.5em}
    	\item O que mais poderíamos descrever com os mesmos conceitos ``abstraídos''? Seria possível descrever uma vaca? Um pássaro? 
	\end{itemize}
	}\par
\end{frame}



% \section{Introdução à programação}
% \section{Tomada de decisões}
% \section{Laços de repetição}
% \section{Estrutura de dados}
% \section{Macros e Funções}
% \section{Estruturas e uniões}
% \section{Alocação da memória}
% \section{Recursividade}

% \part{Conteúdo~da~prova~II}
% \frame{\partpage}

% \section{Pilha}
% \section{Fila}
% \section{Listas}
% \section{Árvores}

% Conceitos Básicos de Linguagem C
% Estrutura de um programa em C;
% Comentários;
% Palavras reservadas;
% Declaração de variáveis e constantes;
% Operadores de atribuição, comparação, aritméticos, lógicos e relacionais;
% Comandos de impressão na tela;
% Comandos de decisão e repetição;
% Vetores;
% Array e Strings;
% Funcionamento e configuração do ambiente para execução;
% Resolução de problemas;
% Conceitos básicos de programação estruturada
% Paradigma procedural;
% Modularização e Principais Vantagens;
% Funções;
% Passagem de parâmetros por valor e por referência;
% Resolução de problemas;
% Recursividade
% Funções recursivas;
% Recursividade direta e indireta;
% Resolução de problemas.



% Algoritmos. Estruturas fundamentais de algoritmos: sequência, tomada de decisão e repetição. Estrutura de Dados. Introdução à linguagem de programação.  A linguagem de programação C. Fundamentos de programação estruturada. Macros e Funções. Vetores, Strings e Matrizes. Estruturas e Uniões. Ponteiros. Funções de entrada e saída. Funções gráficas e textos. Filas, Pilhas, Listas encadeadas. Árvores de busca binária. Práticas laboratoriais. 


% \section{Introdução}

% \begin{frame}{Motivação}
%     \fontsize{12pt}{15.2}\selectfont{
	
%     O estudo desta disciplina faz o aluno adquirir ou aperfeiçoar seu raciocínio lógico no intuito de desenvolverem programas e sistemas em uma determinada linguagem de programação.\\	
% 	\vspace{0.5em}	
	
% 	A Lógica é apresentada como uma técnica eficiente para:}
	
% 	\fontsize{11pt}{12.2}\selectfont{
% 	\begin{itemize}
% 	\vspace{0.5em}
%     	\item a organização de conhecimentos em qualquer área;
%     	\item raciocinar corretamente sem esforço consciente;
%     	\item interpretar e analisar informações rapidamente;
%     	\item aumentar a competência linguística (oral e escrita);
%     	\item adquirir destreza com o raciocínio quantitativo; e
%     	\item detectar padrões em estruturas (premissas, pressuposições, cenários,etc.)
% 	\end{itemize}
% 	}\par
% 	\vspace{1em}
% \end{frame}

% \transblindsvertical

% % A capacidade de pensar de maneira lógica é um dos principais diferenciais para saber como resolver problemas, principalmente na área da computação. Diretamente relacionado a isso, compreender o conceito de algoritmo também é algo fundamental. Com isso em mente, caso seu objetivo seja se tornar um bom programador, o próximo passo é aprender um pseudocódigo, no qual você entrará em contato com detalhes como entrada e saída de dados, assim como o processamento propriamente dito. 


% \begin{frame}{Lógica}
%     \fontsize{12pt}{15.2}\selectfont{
	
% 	\begin{beamercolorbox}[wd=\textwidth]{warning}
%     É a ciência das leis ideais do pensamento e a arte de aplicá-las à pesquisa e à demonstração da verdade.
%     \end{beamercolorbox}
%     }
    
% 	\vspace{0.5em}	
	
% 	\begin{itemize}
% 	\item Deriva do Grego (logos); e
% 	\item Significa:
%     	\fontsize{11pt}{12.2}\selectfont{
%     	\begin{itemize}
%     	\vspace{0.5em}
%         	\item palavra;
%         	\item pensamento;
%         	\item ideia;
%         	\item argumento;
%         	\item relato;
%         	\item razão lógica; 
%         	\item ou princípio lógico.
%     	\end{itemize}}
    	
% 	\end{itemize}
% 	\par
% 	\vspace{1em}
% \end{frame}


% \begin{frame}{Lógica}
%     \fontsize{12pt}{15.2}\selectfont{
	
% 	\begin{beamercolorbox}[wd=\textwidth]{warning}
%     Lógica é o que os lógicos cultivam ou o que está nos tratados de Lógica. 
%     \end{beamercolorbox}
    
    
% 	\vspace{0.5em}	
	
	
%     Não existe uma definição satisfatória de Lógica. Tal questão pertence à Filosofia que trata, entre outras coisas, de temas que não possuem resposta cabal. Esta situação é estranha, pois vamos estudar Lógica sem poder saber exatamente o que ela é ...
    
    
%     }
% 	\par
% 	\vspace{1em}
% \end{frame}


% \begin{frame}{Lógica}
%     \fontsize{12pt}{15.2}\selectfont{
	
% 	A grosso modo, Platão acredita na existência de dois mundos:
    
% 	\vspace{0.5em}	
% 	\begin{itemize}
%     	\item O mundo físico (em que vivemos) e
%     	\item O mundo das entidades ideais.
% 	\end{itemize}
%     \vspace{0.5em}	
    
%     Todas as entidade lógicas estão no mundo das entidades ideais.
%     }
% 	\par
% 	\vspace{1em}
% \end{frame}

% \begin{frame}{Lógica}
%     \fontsize{12pt}{15.2}\selectfont{
	
% 	Ainda de acordo com Platão, o único acesso ao mundo das entidades das ideias é feita através de nosso intelecto, e segundo ele, esta é a razão pela qual poucos o conhecem, e que a nossa relação com tais entidades é de descoberta (e não de criação, por exemplo).

%     }
% 	\par
% 	\vspace{1em}
% \end{frame}


% \begin{frame}{Origem da Lógica}
%     \fontsize{12pt}{15.2}\selectfont{
% 	\begin{itemize}
%     	\item A Lógica teve início na Grécia em 342 a.C.;
%     	\item Aristóteles sistematizou os conhecimentos existentes em Lógica,elevando-a à categoria de ciência;
%     	\item Obra chamada \textit{Organon}(Ferramenta para o correto pensar);
%     	\item Aristóteles preocupava-se com as formas de raciocínio que, a partir de conhecimentos considerados verdadeiros, permitiam obter novos conhecimentos; e 
%     	\item A partir dos conhecimentos tidos como verdadeiros, caberia à Lógica a formulação de leis gerais de encadeamentos lógicos que levariam à descoberta de novas verdades.
% 	\end{itemize}
% 	}\par
% 	\vspace{1em}
% \end{frame}


% \begin{frame}{Argumento Lógica}
%     \fontsize{12pt}{15.2}\selectfont{
% 	\begin{itemize}
%     	\item Em Lógica, o encadeamento de conceitos é chamado de argumento;
%     	\item As afirmações de um argumento são chamadas de proposições;
%     	\item Um argumento é um conjunto de proposições tal que se afirme que uma delas é derivada das demais;
%     	\item Usualmente, a proposição derivada é chamada de conclusão, e as demais, de premissas; e 
%     	\item Em um argumento válido, as premissas são consideradas provas evidentes da verdade da conclusão.
% 	\end{itemize}
% 	}\par
% 	\vspace{1em}
% \end{frame}


% \begin{frame}{Argumento Lógico}
%     \fontsize{14pt}{15.2}\selectfont{
% 	\begin{itemize}
%     	\item Argumento
%     	    \begin{itemize}
%             	\item Se eu estudar, aprenderei (premissa) [Eu estudei]
%             	\item Logo, eu aprendi (conclusão)
%         	\end{itemize}
% 	\end{itemize}
% 	}\par
% 	\vspace{1em}
% \end{frame}


% \begin{frame}{Princípios Lógicos}
%     \fontsize{12pt}{15.2}\selectfont{
    
%     A Lógica Formal repousa sobre três princípios fundamentais que permitem todo seu desenvolvimento posterior, e que dão validade a todos os atos do pensamento e do raciocínio. São eles:
%     \vspace{1em}
%     \fontsize{14pt}{14.2}\selectfont{
% 	\begin{itemize}
%     	\item \textbf{Princípio da Identidade}\\
%         	Afirma A = A e não pode ser B, o que é, é;
%         \item \textbf{Princípio da Não Contradição}\\
%             A = A e nunca pode ser não-A, o que é, é e não pode ser sua negação, ou seja, o ser é, o não ser não é; e
%         \item \textbf{Princípio do Terceiro Excluído}\\
%         Afirma que Ou A é x ou A é y, não existe uma terceira possibilidade
% 	\end{itemize}}
% 	}\par
% 	\vspace{1em}
% \end{frame}



% \begin{frame}{Mito da caverna}
%     \fontsize{12pt}{15.2}\selectfont{
    
%     Segundo Platão (427-347 a.C.), a condição do homem no mundo é semelhante àquela de escravos presos no interior de uma caverna, situação que só lhes permite ver do exterior as sombras que são aí projetadas. A caverna representa o mundo dos sentidos, no qual só se percebem as sombras das coisas. O exterior é o mundo das ideias, representado pelas próprias coisas e pelo Sol, que simboliza o Bem. De acordo com Platão, a Filosofia é que dá ao homem a condição de sair da caverna e perceber a realidade e o mundo das ideias.
    
% 	}\par
% 	\vspace{1em}
% \end{frame}






% \begin{frame}{Alegoria da caverna}
%     \fontsize{14pt}{15.2}\selectfont{
% 	\begin{figure}[h]
%     \centering
%     \vspace{0.5cm}
%     \includegraphics[width=0.9\textwidth]{images/alegoria-caverna.jpg}
%     \end{figure}
% 	}\par
% 	\vspace{1em}
% \end{frame}

