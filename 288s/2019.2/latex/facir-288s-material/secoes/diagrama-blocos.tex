\section{Diagrama de Blocos}


\begin{frame}{Fluxograma}
    \fontsize{12pt}{15.2}\selectfont{
	Ferramenta usada e pelos profissionais da área de análise de sistemas.
	\begin{itemize}
	\vspace{1em}
	\item Tem por objetivo descrever o fluxo de ação de um determinado trabalho lógico, seja manual ou mecânico, especificando os suportes usados para os dados e para as informações.
	\item Pode ser feito em qualquer nível de abstração.
	\item Utiliza símbolos convencionais (norma ISO 5807:1985) e permite poucas variações. São desenhos geométricos básicos.
	\end{itemize}
	}\par
	\vspace{1em}
\end{frame}

\begin{frame}{Diagrama de Bloco}
    \fontsize{12pt}{15.2}\selectfont{
	Também conhecido como diagrama de fluxo (não confundir com fluxograma), é uma ferramenta usada pelo profissional de desenvolvimento de programas, seja ele o programador e afins.
	\begin{itemize}
	\vspace{1em}
	\item Tem por objetivo descrever o método e sequência de ações a serem estabelecidas para um computador.
	\item Pode ser feito em qualquer nível de abstração.
	\item Utiliza símbolos geométricos, os quais estabelecerão as sequências de operações a serem efetuadas em processamento computacional.
	\end{itemize}
	}\par
	\vspace{1em}
\end{frame}


\begin{frame}{Raciocínio lógico}
    \fontsize{24pt}{15.2}\selectfont{
	Diagrama de Bloco + Codificação
	}\par
	\vspace{1em}
	\fontsize{12pt}{15.2}\selectfont{
	Modo de solucionar um problema por meio um algoritmo.
	}\par
\end{frame}

\begin{frame}{Definições Básicas}
    \fontsize{12pt}{15}\selectfont{
	Segundo \cite{2008:manzano}, muitas pessoas gostam de falar ou julgar que possuem e sabem usar o \textbf{raciocínio lógico}, porém quando questionadas direta ou indiretamente, perdem essa linha de raciocínio, pois ele depende de inúmeros fatores para completá-lo, tais como: calma, conhecimento, vivência, versatilidade, experiência, criatividade, ponderação, responsabilidade, autodisciplina, entre outros.
	}\par
	\vspace{1em}
\end{frame}

\begin{frame}{Definições Básicas}
    \fontsize{12pt}{15}\selectfont{
	Para usar a lógica, é necessário ter domínio sobre o pensamento, bem como saber pensar, ou seja, possui a ``Arte do pensar''. Alguns definem o raciocínio lógico como um conjunto de estudos que visa determinar os processos intelectuais que são as condições gerais do conhecimento verdadeiro. Outros preferem dizer que é a sequência coerente, regular e necessária de acontecimentos, de coisas ou fatos, ou até mesmo, que é a maneira do raciocínio particular que cabe a um indivíduo ou a um grupo.
	}\par
	\vspace{1em}
\end{frame}


\begin{frame}{Definições Básicas}
    \fontsize{12pt}{15}\selectfont{
	Logo, pode-se entende que lógica é a ciência que estuda as leis e critérios de validade que regem o pensamento e a demostração, ou seja, ciência dos princípios formais do raciocínio.
	}\par
	\vspace{1em}
\end{frame}


\begin{frame}{Necessidade de usar a lógica}
    \fontsize{12pt}{15}\selectfont{
	Usar a lógica é um fator a ser considerado por todos, principalmente pelos profissionais da área da Tecnologia de Informação (programadores, analistas de sistemas e suporte), pois seu dia-a-dia dentro das organizações é solucionar problemas e atingir os objetivos apresentados por seus usuários com eficiência e eficácia, utilizando recursos computacionais e/ou automatizados mecatronicamente. Saber lidar com controle, de planejamento e de estratégia requer atenção e boa performance de conhecimento de nosso raciocínio. Porém, é necessário considerar que ninguém ensina ninguém a pensar, pois todas as pessoas normais possuem esse ``dom''. 
	
	
	}\par
	\vspace{1em}
\end{frame}

\begin{frame}{Necessidade de usar a lógica}
    \fontsize{12pt}{15}\selectfont{
	O objetivo aqui é mostrar como desenvolver e aperfeiçoar melhor essa técnica, lembrando que para isso você deve ser persistente e praticá-la constantemente, chegando à exaustão sempre que julgar necessário. 
	}\par
	\vspace{1em}
\end{frame}


\begin{frame}{Aplicabilidade da lógica no auxílio do desenvolvimento de programas}
    \fontsize{12pt}{15}\selectfont{
	É comum que profissionais da área de Tecnologia da Informação (desenvolvimento) prefiram preparar um programa iniciando o seu projeto com um para demonstrar de forma concreta sua linha de raciocínio lógico (que é um elemento abstrato). 
	\vspace{0.5cm}
	
	O diagrama de blocos é um instrumento que visa estabelecer visualmente a sequência de operações a ser efetuada por um programa de computador. 
	}\par
	\vspace{1em}
\end{frame}

\begin{frame}{Aplicabilidade da lógica no auxílio do desenvolvimento de programas}
    \fontsize{12pt}{15}\selectfont{
	A técnica de desenvolvimento de diagrama de blocos permite ao desenvolvimento uma grande facilidade na posterior codificação do programa em qualquer uma das linguagens de programação existentes, pois elaboração não se leva em que uma linguagem utiliza. 
	\vspace{0.5cm}
	
	O diagrama de blocos é uma ferramenta que possibilita definir o detalhamento operacional que um programa deve executar sendo um instrumento tão valioso quanto é uma planta para um arquiteto.
	
	}\par
	\vspace{1em}
\end{frame}


\begin{frame}{Aplicabilidade da lógica no auxílio do desenvolvimento de programas}
    \fontsize{12pt}{15}\selectfont{
	
    A técnica mais importante no projeto da logica de programas denomina-se programação estruturada, a qual consiste em uma metodologia de projeto, objetivando: 
    
    \begin{itemize}
        \item Agilizar a codificação da escrita da programação; 
        \item Facilitar a depuração da sua leitura; 
        \item Permitir a verificação de possíveis falhas apresentadas pelo programas;
        \item Facilitar as alterações e atualizações dos programa.
    \end{itemize}
        
        
	}\par
	\vspace{1em}
\end{frame}


\begin{frame}{Aplicabilidade da lógica no auxílio do desenvolvimento de programas}
    \fontsize{12pt}{15}\selectfont{
    E deve ser composta por quatro passos fundamentais:
    \begin{itemize}
        \item Escrever as instruções em sequências ligadas entre si apenas por estruturas sequenciais, repetitivas ou de selecionamento;
        \item Escrever instruções em grupos pequenos e combiná-las;
        \item Distribuir módulos do programa entre os diferentes programadores que trabalharão sob a supervisão de um programador sênio, ou chefe de programação;
        \item Revisar o trabalho executado em reuniões regulares e previamente programadas, em que compareçam apenas programadores de um mesmo nível.
    \end{itemize}
        
	}\par
	\vspace{1em}
\end{frame}




% \begin{frame}{Símbolos básicos}
%     \fontsize{8pt}{10}\selectfont{
%     \vspace{0.3cm}

%         \begin{tikzpicture}[scale=1]
%         \matrix[nodes={draw, ultra thick, fill=none}, row sep=0.3cm,column sep=0.5cm]{
%         \node[ellipse] {Terminal}; &&
%         \node[draw,align=left,line width=0.2mm] at (0,0) {utilizado como ponto para indicar o início e/ou fim do \\fluxo de um programa.};\\
%         };
%         \end{tikzpicture} 
        
%         \begin{tikzpicture}[scale=1]
%         \matrix[nodes={draw, ultra thick, fill=none}, row sep=0.3cm,column sep=0.5cm]{
%         \node[rectangle] {Processamento}; &&
%         \node[draw,align=left,line width=0.2mm] at (0,0) {Utilizado para indicar cálculos (algoritmos) a efetuar, atribuições de\\ valores ou qualquer manipulação de dados que tenha um bloco\\ específico para sua descrição.};\\
%         };
%         \end{tikzpicture} 


%         \begin{tikzpicture}[scale=1]
%         \matrix[nodes={draw, ultra thick, fill=none}, row sep=0.3cm,column sep=0.5cm]{
%         \draw[fill=none,line width=0.5mm] (0,0) -- (2,0) -- ++(60:0.5) -- ++(-2,0) -- cycle; &&
%         \node[draw,align=left,line width=0.2mm] at (0,0) {Entrada e saída de dados - qualquer dispositivo de entrada\\ ou saída de addos.};\\
%         };
%         \end{tikzpicture}


%         \begin{tikzpicture}[scale=1]
%         \matrix[nodes={draw, ultra thick, fill=none}, row sep=0.3cm,column sep=0.5cm]{
%         \node[diamond] {Decisão}; &&
%         \node[draw,align=left,line width=0.2mm] at (0,0) {Indica a decisão que deve ser tomada, indicando a possibilidade de desvios\\ para diversos outros pontos do fluxo, dependendo do resultado de\\ comparação e de acordo com situações variáveis.};\\
%         };
%         \end{tikzpicture} 
        
%         \begin{tikzpicture}[scale=1]
%         \matrix[nodes={draw, ultra thick, fill=none}, row sep=0.3cm,column sep=0.5cm]{
%         \draw [->,>=stealth,line width=0.5mm] (0,.5) -- (2,.5); &&
%         \node[draw,align=left,line width=0.2mm] at (0,0) {Seta de fluxo de dados - permite indicar o sentido do fluxo de dados. \\Serve exclusivamente para conectar os símbolos ou blocos existentes};\\
%         }
%         \end{tikzpicture} 
        
% 	}\par
% 	\vspace{1em}
% \end{frame}

